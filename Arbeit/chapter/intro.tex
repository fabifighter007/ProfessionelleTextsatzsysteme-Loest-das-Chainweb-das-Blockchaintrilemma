\section{Einleitung}
Blockchaintechnologien ermöglichen es, Vermögen dezentral zu verwalten oder zu versenden. Zusätzlich schafft sie Transparenz, denn jede Transaktion ist öffentlich einsehbar. Über die letzten Jahre haben sich durch mehr Nutzer neue Herausforderungen an die junge Technologie gestellt. Eine Blockchain kann nicht beliebig viele Transaktionen gleichzeitig verarbeiten, denn jeder Block ist auf eine bestimmte Größe limitiert. Mit steigendem Transaktionsaufkommen und gleich bleibendem Angebot an vorhandenem Platz in jedem neuen Block steigt schlussendlich der Preis pro Transaktion. Um das Problem zu lösen, wurden neue Technologien entwickelt: Diese reichen von anderen Konsensfindungsverfahren (Proof of Stake $POS$, delegiertes  Proof-of-Stake $dPOS$) bis hin zu Layer 2 Lösungen oder Quantentechnologie. \cite{BundesamtfurSicherheitinderInformationstechnik.2019}
\\
In dieser Arbeit wird die neue Blockchaintechnologie des Chainwebs betrachtet und bewertet. Es wird die Funktionsweise untersucht und sodann experimentell gezeigt, wie das Chainweb eine Proof of Work Blockchain skaliert. Hierbei steht der Aspekt der Sicherheit eine wesentliche Rolle.