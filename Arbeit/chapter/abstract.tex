\begin{abstract}
    Das  Chainweb  möchte  das  Blockchaintrilemma  gelöst  haben.  Dieses besagt,  dass  eine  Blockchain  nur  zwei  dreier  wichtiger  Eigenschaften vereinen kann: Dezentralität, Sicherheit und Skalierbarkeit. Dank  beliebig vieler,  unabhängiger  Ketten  kann  das  Chainweb  effektiver als bisherige  Proof of Work  Blockchains  arbeiten. In dieser Arbeit wird das Chainweb hinsichtlich des Blockchaintrilemmas bewertet und mit anderen Proof of Work Blockchains vergleichen. Es wird dargestellt, wie eine POW-Blockchain auf Layer 1 skalieren kann und erläutert, wie die Sicherheit der Blockchain davon profitiert. Der Punkt der Dezentralität wird untersucht und gezeigt, dass ein Akteur zu viel Rechenleistung im Netzwerk besitzt. \\
\end{abstract}

\begin{IEEEkeywords}
	Blockchain, Chainweb, Blockchaintrilemma, Proof of Work
\end{IEEEkeywords}